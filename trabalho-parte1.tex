\documentclass[12pt, a4paper, oneside]{abntex2}

% --- PACOTES UTILIZADOS ---
\usepackage[utf8]{inputenc}
\usepackage[T1]{fontenc}
\usepackage[brazil]{babel}
\usepackage{graphicx}
\usepackage{lipsum}
\usepackage{xcolor}
\usepackage{amsmath, amssymb}
\usepackage{array}
\usepackage{longtable}
\usepackage{geometry}

% hyperref é carregado por abntex2, configurar opções com \hypersetup
% --- CONFIGURAÇÃO DAS MARGENS (Padrão ABNT) ---
\geometry{
    a4paper,
    left=3cm,
    right=2cm,
    top=3cm,
    bottom=2cm
}

% Configuração do hyperref (carregado pela classe abntex2)
\hypersetup{hidelinks}

% --- INFORMAÇÕES DO TRABALHO (Para Capa e Folha de Rosto) ---
\titulo{Trabalho Parte I – Requisitos de IHC}
\preambulo{Relatório de Pesquisa com Usuários e Documento de Especificação dos Requisitos do sistema de Apoio à Telemedicina}
\autor{Felipe Lazzarini Cunha \\ Gabriel Campos Lima Alves}
\instituicao{%
  Universidade Federal de Juiz de Fora -- UFJF
  \par
  Instituto de Ciências Exatas
  \par
  Departamento de Ciência da Computação}
% \logo{\includegraphics[width=3cm]{Imagens/ufjf-logo.png}}
\makeatletter
\providecommand{\disciplina}[1]{\gdef\@disciplina{#1}}
\makeatother
\disciplina{DCC174 -- Interação Humano-Computador}
\local{Juiz de Fora}
\data{Junho de 2025} % Conforme data de entrega

% --- INÍCIO DO DOCUMENTO ---
\begin{document}

% --- ELEMENTOS PRÉ-TEXTUAIS ---
% Substituindo \imprimircapa por uma capa personalizada
\begin{capa}
    \centering
    \begin{minipage}{0.2\textwidth}
        \includegraphics[width=3cm]{Imagens/ufjf-logo.png}
    \end{minipage}
    \hfill
    \begin{minipage}{0.7\textwidth}
        \raggedright
        \textbf{UNIVERSIDADE FEDERAL DE JUIZ DE FORA} \\
        \textbf{INSTITUTO DE CIÊNCIAS EXATAS} \\
        \textbf{DEPARTAMENTO DE CIÊNCIA DA COMPUTAÇÃO (DCC)}
    \end{minipage}
    \vspace*{3cm}
    \hrule
    \vspace*{5cm}
    \textbf{\Large DCC174 Interação Humano-Computador} \\
    \vspace{0.5cm}
    \textbf{\Large Trabalho Parte I – Requisitos de IHC} \\
    \vspace{3cm}
    Felipe Lazzarini Cunha \\
    Gabriel Campos Lima Alves \\
    \vfill
    \vspace*{1cm}
    \begin{center}
    Juiz de Fora\\
    Junho de 2025
    \end{center}
\end{capa}

% Customizando a folha de rosto
% \imprimirfolhaderosto
\begin{folhaderosto}
    \centering
    \begin{minipage}{0.2\textwidth}
        \includegraphics[width=3cm]{Imagens/ufjf-logo.png}
    \end{minipage}
    \hfill
    \begin{minipage}{0.7\textwidth}
        \raggedright
        \textbf{UNIVERSIDADE FEDERAL DE JUIZ DE FORA} \\
        \textbf{INSTITUTO DE CIÊNCIAS EXATAS} \\
        \textbf{DEPARTAMENTO DE CIÊNCIA DA COMPUTAÇÃO (DCC)}
    \end{minipage}
    \vspace*{3cm}
    \rule{\textwidth}{0pt}

    \textbf{\Large DCC174 Interação Humano-Computador} \\
    \vspace{0.5cm}
    \textbf{\Large Trabalho Parte I – Requisitos de IHC} \\
    \vspace{3cm}
    Felipe Lazzarini Cunha \\
    Gabriel Campos Lima Alves \\
    \vspace*{3cm}
    \hfill % Empurra o minipage para a direita
    \begin{minipage}{0.6\textwidth} % Ajuste a largura conforme necessário
        \raggedright % Para alinhar o texto à esquerda dentro da "caixa"
        Relatório de Pesquisa com Usuários e Documento de Especificação dos Requisitos do sistema de Apoio à Telemedicina
    \end{minipage}
    \vfill
    \vspace*{1cm}
    \begin{center}
    Juiz de Fora\\
    Junho de 2025
    \end{center}
\end{folhaderosto}

% --- SUMÁRIO (Gerado automaticamente) ---
\pdfbookmark[0]{\contentsname}{toc}
\tableofcontents
\cleardoublepage

% --- INÍCIO DOS ELEMENTOS TEXTUAIS ---
\textual

% =========================================================================
% PARTE 1: PESQUISA COM USUÁRIOS
% Conforme enunciado do trabalho e modelo de sumário.
% =========================================================================
\chapter{Pesquisa com Usuários}

\section{Planejamento}
\label{sec:planejamento}
% Descrever aqui os objetivos da pesquisa, os métodos que vocês adotaram
\lipsum[1]

\subsection{Objetivos da Pesquisa}
\lipsum[2]

\subsection{Métodos Adotados e Justificativa}
\lipsum[3]

\section{Identificação dos Stakeholders}
\label{sec:stakeholders}
% Listar e descrever todos os stakeholders identificados.
\lipsum[4]

\section{Identificação dos Papis}
\label{sec:papeis}
% Descrever os papéis primários (interagem diretamente com o sistema) e secundários
% (interagem por meio de outra pessoa). Pensar no papel do
% médico generalista que registra a dúvida, o docente especialista que responde, etc.
\lipsum[5]

\section{Dados Coletados}
\label{sec:dados_coletados}
% Apresentar e analisar os dados que vocês coletaram na pesquisa.
% Incluir os perfis de usuário e personas que vocês criaram a partir dos dados.
% Lembrar de caracterizar os usuários com base nos critérios do documento.
\lipsum[6]


% =========================================================================
% PARTE 2: DOCUMENTO DE REQUISITOS
% =========================================================================
\chapter{Documento de Requisitos}

\section{Escopo do Sistema}
\label{sec:escopo}
% Descrever aqui os objetivos do sistema. Utilizar a descrição detalhada do
% "Tema 1 – Aplicativo de Apoio à Telemedicina" fornecida no enunciado do
% trabalho.
% A plataforma alvo é Mobile.
\lipsum[7]

\section{Metas de Design}
\label{sec:metas_design}
% Descrever os objetivos dos usuários que o sistema deve apoiar e os
% critérios de qualidade de uso (Usabilidade, Acessibilidade, etc.).
% Exemplo: "Aumentar a eficiência das atividades dos usuários".
% Exemplo: "Comunicar adequadamente, através da interface, a visão do designer".
\lipsum[8]

\section{Caracterização dos Usuários e dos Stakeholders}
\label{sec:caracterizacao_usuarios}
% Detalhar aqui as características dos usuários e stakeholders identificados
% na Seção \ref{sec:stakeholders}. Utilizar o quadro resumo das informações que afetam o uso.
% Criar perfis e/ou personas, conforme solicitado no enunciado.
\lipsum[9]

\section{Requisitos Funcionais}
\label{sec:requisitos_funcionais}
% Especificar cada requisito funcional (funcionalidade visível ao usuário)  
% Usar o template da Tabela 1 do modelo.
% Um requisito deve ter identificador, nome, stakeholders, pré-condições,
% descrição e pós-condições.

\lipsum[10]

% --- Exemplo de tabela para Requisito Funcional ---
% Use 'longtable' para que a tabela possa quebrar entre páginas se for muito grande.
\begin{longtable}{|p{0.25\linewidth}|p{0.7\linewidth}|}
    \caption{Exemplo de especificação de requisito funcional.} \label{tab:rf01_exemplo} \\
    \hline
    \multicolumn{2}{|l|}{\textbf{RF01: Efetuar Login}} \\
    \hline
    \endfirsthead
    \multicolumn{2}{c}%
    {{\bfseries\tablename\ \thetable{} -- continuação}} \\
    \hline
    \multicolumn{2}{|l|}{\textbf{RF01: Efetuar Login}} \\
    \hline
    \endhead
    \hline \multicolumn{2}{r}{{\textit{Continua na próxima página}}} \\
    \endfoot
    \hline
    \endlastfoot

    % Conteúdo da tabela
    \textbf{Interessado(s)/ Stakeholder(s):} & Usuário (Médico de UBS, Docente, Estudante de Medicina). \\
    \hline
    \textbf{Pré-condição(ões):} & O usuário deve possuir um cadastro ativo no sistema. \\
    \hline
    \textbf{Descrição:} & O usuário deverá efetuar o login no aplicativo informando suas credenciais (ex: e-mail ou CRM) e senha. Após informar os dados, o usuário deve pressionar o botão "Entrar". \\
    \hline
    \textbf{Pós-condição(ões):} & Caso o usuário informe credenciais válidas, o sistema fornece acesso à sua área restrita, com as funcionalidades correspondentes ao seu perfil. Caso contrário, o sistema informa uma mensagem de erro. \\
    \hline
\end{longtable}

% Adicione outras tabelas para os demais requisitos funcionais aqui.
% Ex: RF02: Registrar Dúvida, RF03: Enviar Dúvida, RF04: Responder Dúvida.

\section{Requisitos Não Funcionais}
\label{sec:requisitos_nao_funcionais}
% Descrever os requisitos não funcionais, que são restrições aos serviços ou
% funcionalidades.
% Exemplo: "o sistema deverá fornecer uma resposta ao usuário em até 2 segundos
% após uma tentativa de login".
% Pode incluir requisitos de portabilidade, tecnologia, usabilidade, segurança, etc.
\lipsum[11]

\section{Restrições do Sistema}
\label{sec:restricoes}
% Especificar as restrições tecnológicas, de compatibilidade com plataformas,
% interoperabilidade com outros sistemas, etc..
\lipsum[12]


% --- ELEMENTOS PÓS-TEXTUAIS (Referências, Apêndices, etc.) ---
\postextual

% --- BIBLIOGRAFIA ---
\bibliography{referencias} % Cria um arquivo 'referencias.bib' para gerenciar suas fontes.

\end{document}