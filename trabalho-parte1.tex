\documentclass[12pt, a4paper, oneside]{abntex2}

% --- PACOTES UTILIZADOS ---
\usepackage[utf8]{inputenc}
\usepackage[T1]{fontenc}
\usepackage[brazil]{babel}
\usepackage{graphicx}
\usepackage{amsmath, amssymb}
\usepackage{array}
\usepackage{longtable}
\usepackage{geometry}

% --- CONFIGURAÇÃO DAS MARGENS (Padrão ABNT) ---
\geometry{
    a4paper,
    left=3cm,
    right=2cm,
    top=3cm,
    bottom=2cm
}

% Configuração do hyperref (carregado pela classe abntex2)
\hypersetup{hidelinks}

% --- INFORMAÇÕES DO TRABALHO (Para Capa e Folha de Rosto) ---
\titulo{Trabalho Parte I – Requisitos de IHC}
\preambulo{Relatório de Pesquisa com Usuários e Documento de Especificação dos Requisitos do sistema de Apoio à Telemedicina}
\autor{Felipe Lazzarini Cunha \\ Gabriel Campos Lima Alves}
\instituicao{%
  Universidade Federal de Juiz de Fora -- UFJF
  \par
  Instituto de Ciências Exatas
  \par
  Departamento de Ciência da Computação}
\makeatletter
\providecommand{\disciplina}[1]{\gdef\@disciplina{#1}}
\makeatother
\disciplina{DCC174 -- Interação Humano-Computador}
\local{Juiz de Fora}
\data{Junho de 2025}

% --- INÍCIO DO DOCUMENTO ---
\begin{document}

% --- ELEMENTOS PRÉ-TEXTUAIS ---
\begin{capa}
    \centering
    \begin{minipage}{0.2\textwidth}
        \includegraphics[width=3cm]{Imagens/ufjf-logo.png}
    \end{minipage}
    \hfill
    \begin{minipage}{0.7\textwidth}
        \raggedright
        \textbf{UNIVERSIDADE FEDERAL DE JUIZ DE FORA} \\
        \textbf{INSTITUTO DE CIÊNCIAS EXATAS} \\
        \textbf{DEPARTAMENTO DE CIÊNCIA DA COMPUTAÇÃO (DCC)}
    \end{minipage}
    \vspace*{3cm}
    \hrule
    \vspace*{5cm}
    \textbf{\Large DCC174 Interação Humano-Computador} \\
    \vspace{0.5cm}
    \textbf{\Large Trabalho Parte I – Requisitos de IHC} \\
    \vspace{3cm}
    Felipe Lazzarini Cunha \\
    Gabriel Campos Lima Alves \\
    \vfill
    \vspace*{1cm}
    \begin{center}
    Juiz de Fora\\
    Junho de 2025
    \end{center}
\end{capa}

\begin{folhaderosto}
    \centering
    \begin{minipage}{0.2\textwidth}
        \includegraphics[width=3cm]{Imagens/ufjf-logo.png}
    \end{minipage}
    \hfill
    \begin{minipage}{0.7\textwidth}
        \raggedright
        \textbf{UNIVERSIDADE FEDERAL DE JUIZ DE FORA} \\
        \textbf{INSTITUTO DE CIÊNCIAS EXATAS} \\
        \textbf{DEPARTAMENTO DE CIÊNCIA DA COMPUTAÇÃO (DCC)}
    \end{minipage}
    \vspace*{3cm}
    \rule{\textwidth}{0pt}
    \textbf{\Large DCC174 Interação Humano-Computador} \\
    \vspace{0.5cm}
    \textbf{\Large Trabalho Parte I – Requisitos de IHC} \\
    \vspace{3cm}
    Felipe Lazzarini Cunha \\
    Gabriel Campos Lima Alves \\
    \vspace*{3cm}
    \hfill
    \begin{minipage}{0.6\textwidth}
        \raggedright
        Relatório de Pesquisa com Usuários e Documento de Especificação dos Requisitos do sistema de Apoio à Telemedicina
    \end{minipage}
    \vfill
    \vspace*{1cm}
    \begin{center}
    Juiz de Fora\\
    Junho de 2025
    \end{center}
\end{folhaderosto}

% --- SUMÁRIO ---
\pdfbookmark[0]{\contentsname}{toc}
\tableofcontents
\cleardoublepage

% --- INÍCIO DOS ELEMENTOS TEXTUAIS ---
\textual

% =========================================================================
% PARTE 1: PESQUISA COM USUÁRIOS
% =========================================================================
\chapter{Pesquisa com Usuários}

% Apresentando o objetivo geral da pesquisa
Este relatório especifica o processo de pesquisa com usuários (UX Research) conduzido para o desenvolvimento de um aplicativo de apoio à telemedicina. O objetivo foi caracterizar os usuários do sistema e identificar suas necessidades, fundamentando o projeto de design na abordagem da Engenharia Cognitiva.

\section{Planejamento}
\label{sec:planejamento}

\subsection{Objetivos da Pesquisa}
O planejamento da nossa pesquisa foi norteado pelos seguintes objetivos centrais:
\begin{itemize}
    \item \textbf{Compreender o fluxo atual:} Esquematizar como os médicos generalistas e Unidades Básicas de Saúde (UBSs) solicitam e recebem apoio de especialistas atualmente, encontrando os principais gargalos e ineficiências do processo que envolve deslocamento físico.
    \item \textbf{Identificar necessidades de informação:} Determinar os dados e informações fundamentais para que uma dúvida clínica seja registrada de forma completa e respondida eficazmente por um especialista.
    \item \textbf{Caracterizar os usuários:} Levantar o perfil dos três principais grupos de usuários (médicos de UBS, docentes especialistas e estudantes de medicina), incluindo suas habilidades técnicas, rotinas de trabalho e limitações.
    \item \textbf{Mapear dores e objetivos:} Identificar as principais frustrações (dores) e os objetivos que cada grupo de usuários espera alcançar com o uso de uma nova ferramenta digital.
    \item \textbf{Avaliar critérios de qualidade:} Identificar quais atributos em termos de facilidade de uso, como eficiência, segurança na prevenção de erros e clareza, são prioritários para os usuários.
\end{itemize}

\subsection{Métodos Adotados e Justificativa}
Para atingir nossos objetivos, usamos uma abordagem combinada, misturando métodos quantitativos e qualitativos:
\begin{itemize}
    \item \textbf{Método Adotado:}
    \begin{itemize}
        \item Questionário Online: Desenvolvemos e distribuímos um questionário para um grupo amplo de médicos e estudantes de medicina.
        \item Entrevista Semi-estruturada: Realizamos entrevistas em profundidade com dois médicos de UBS, um docente especialista e dois estudantes em estágio.
    \end{itemize}
    \item \textbf{Justificativa:}
    \begin{itemize}
        \item O questionário foi escolhido por sua capacidade de coletar dados quantitativos de um número maior de participantes de forma rápida, permitindo-nos identificar padrões e validar hipóteses sobre o uso de tecnologia e dificuldades comuns.
        \item As entrevistas foram essenciais para aprofundar os dados coletados no questionário. Este método qualitativo nos permitiu explorar o contexto, as motivações e as ``dores'' dos usuários de uma maneira que dados numéricos não conseguem capturar, sendo fundamental para a construção de personas realistas e para entender o modelo mental dos usuários, um pilar da Engenharia Cognitiva.
    \end{itemize}
\end{itemize}

% Incluindo os modelos dos instrumentos utilizados
\textbf{Modelos de Questionário e Formulário de Entrevista:}
\begin{itemize}
    \item \textbf{Modelo de Questionário (Resumido):}
    \begin{itemize}
        \item Seção 1: Perfil do Participante (Cargo, Especialidade, Tempo de Experiência, UBS/Universidade de atuação).
        \item Seção 2: Processo Atual (``Como você geralmente sana dúvidas clínicas complexas?'', ``Quais ferramentas você utiliza para se comunicar com especialistas?'', ``Com que frequência você precisa de uma segunda opinião?'').
        \item Seção 3: Tecnologia (``Quão confortável você se sente usando aplicativos em seu dia a dia?'', ``Qual smartphone você utiliza?'', ``Quais preocupações você tem sobre o uso de tecnologia para discutir casos de pacientes?'').
        \item Seção 4: Necessidades (``Quais informações são indispensáveis ao enviar uma dúvida?'', ``O que tornaria o processo de consulta a um especialista mais eficiente?'').
    \end{itemize}
    \item \textbf{Roteiro de Entrevista (Tópicos Principais):}
    \begin{itemize}
        \item Abertura: Apresentação e explicação dos objetivos da pesquisa.
        \item Rotina de Trabalho: ``Descreva um dia típico de trabalho para você.''
        \item Estudo de Caso: ``Pode me contar sobre a última vez que você teve uma dúvida complexa sobre um paciente? Como você resolveu?''
        \item Colaboração: (Para docentes e estudantes) ``Como funciona a colaboração entre vocês para discutir casos clínicos?''
        \item Expectativas: ``Se você tivesse um aplicativo para isso, o que ele precisaria fazer para realmente te ajudar?''
    \end{itemize}
\end{itemize}

\section{Identificação dos Stakeholders}
\label{sec:stakeholders}
Através da pesquisa, identificamos os seguintes stakeholders:
\begin{itemize}
    \item \textbf{Usuários Primários:}
    \begin{itemize}
        \item Médico Generalista (UBS)
        \item Docente Especialista
        \item Estudante de Medicina
    \end{itemize}
    \item \textbf{Usuários Secundários:}
    \begin{itemize}
        \item Paciente
    \end{itemize}
    \item \textbf{Outros Stakeholders:}
    \begin{itemize}
        \item Gestores da Saúde (UBS e Universidade)
    \end{itemize}
\end{itemize}

\section{Identificação dos Papéis}
\label{sec:papeis}
Os papéis dos usuários no sistema são:
\begin{itemize}
    \item \textbf{Médico Generalista (UBS):} Interage diretamente com o sistema para registrar e enviar dúvidas clínicas, acompanhar o status e receber respostas.
    \item \textbf{Docente Especialista:} Interage para receber, analisar e responder às dúvidas, em colaboração com os estudantes.
    \item \textbf{Estudante de Medicina:} Interage para visualizar as dúvidas e auxiliar o docente na elaboração da resposta.
    \item \textbf{Paciente:} Embora não interaja diretamente com o sistema, é o principal beneficiado com a melhoria da qualidade e agilidade do diagnóstico.
\end{itemize}

\section{Dados Coletados}
\label{sec:dados_coletados}
A análise dos dados coletados revelou os seguintes insights:
\begin{itemize}
    \item \textbf{Dados Quantitativos (do Questionário):}
    \begin{itemize}
        \item 85\% dos médicos de UBS relataram usar aplicativos de mensagens pessoais (como WhatsApp) para discutir casos, mas 95\% deles expressaram preocupação com a segurança e a falta de formalização.
        \item A dificuldade mais citada (70\% dos respondentes) para obter uma segunda opinião é o tempo de espera e a dificuldade de contatar o especialista certo.
    \end{itemize}
    \item \textbf{Dados Qualitativos (das Entrevistas):}
    \begin{itemize}
        \item Uma dor recorrente entre os especialistas é o recebimento de informações incompletas, o que exige múltiplos contatos para esclarecimento.
        \item Os estudantes sentem que poderiam aprender mais se o processo de discussão de casos fosse mais estruturado e documentado.
        \item Médicos de UBS valorizam a autonomia, mas sentem-se inseguros em tomar decisões complexas sem um respaldo rápido e confiável.
    \end{itemize}
\end{itemize}

% Especificando perfis de usuário e personas
\subsection{Especificação de Perfis de Usuário e Personas}
Com base na análise dos dados, criamos os seguintes perfis e personas:

\textbf{Perfil de Usuário 1: Médico Generalista de UBS}
\begin{itemize}
    \item \textbf{Características:} Profissional com alta carga de trabalho, atuando na linha de frente do sistema de saúde. Possui bom conhecimento clínico geral, mas necessita de apoio em casos especializados. Habilidade com tecnologia é variável, mas utiliza smartphones diariamente para comunicação.
    \item \textbf{Necessidades:} Respostas rápidas e confiáveis, comunicação segura, facilidade para registrar e acompanhar suas dúvidas.
\end{itemize}

\textbf{Perfil de Usuário 2: Docente Especialista}
\begin{itemize}
    \item \textbf{Características:} Médico com vasta experiência em uma especialidade (ex: Cardiologia, Dermatologia). Atua na formação de novos médicos. Agenda extremamente ocupada, dividida entre atividades clínicas, acadêmicas e de pesquisa.
    \item \textbf{Necessidades:} Receber casos bem documentados, otimizar o tempo de resposta, utilizar os casos como ferramenta de ensino para os estagiários.
\end{itemize}

\textbf{Persona 1: Dr. Ana Souza (Médica Generalista)}
\begin{itemize}
    \item \textbf{Quem é ela:} 35 anos, médica de família em uma UBS de um município próximo a Juiz de Fora há 5 anos. É dedicada aos seus pacientes, mas se sente sobrecarregada com o volume de atendimentos.
    \item \textbf{Objetivos:}
    \begin{itemize}
        \item Aumentar a precisão de seus diagnósticos.
        \item Obter respostas de especialistas de forma rápida e segura.
        \item Reduzir a necessidade de encaminhar pacientes para outras cidades.
    \end{itemize}
    \item \textbf{Frustrações:}
    \begin{itemize}
        \item ``Perco muito tempo tentando ligar para colegas especialistas que quase nunca podem atender.''
        \item ``Sinto-me insegura ao usar meu WhatsApp pessoal para enviar dados de pacientes.''
        \item ``Às vezes, a resposta demora tanto que o quadro clínico do paciente já mudou.''
    \end{itemize}
\end{itemize}

\textbf{Persona 2: Dr. Ricardo Borges (Docente Especialista)}
\begin{itemize}
    \item \textbf{Quem é ele:} 55 anos, cardiologista e professor na Faculdade de Medicina da UFJF. É apaixonado por ensinar e supervisiona um grupo de estagiários na UBS.
    \item \textbf{Objetivos:}
    \begin{itemize}
        \item Utilizar casos reais para o ensino prático dos estudantes.
        \item Fornecer orientação de alta qualidade para os colegas da atenção básica.
        \item Otimizar seu tempo, respondendo às dúvidas de forma assíncrona.
    \end{itemize}
    \item \textbf{Frustrações:}
    \begin{itemize}
        \item ``Recebo pedidos de ajuda com pouquíssimos detalhes, o que me obriga a fazer várias perguntas antes de poder ajudar.''
        \item ``É difícil acompanhar o desenvolvimento dos casos que oriento à distância.''
        \item ``Gostaria que os estagiários participassem mais ativamente da elaboração das respostas.''
    \end{itemize}
\end{itemize}

% =========================================================================
% PARTE 2: DOCUMENTO DE REQUISITOS
% =========================================================================
\chapter{Documento de Requisitos}

% Introduzindo o objetivo do documento de requisitos
Este documento traduz os achados da pesquisa em uma especificação formal dos requisitos do Aplicativo de Apoio à Telemedicina.

\section{Escopo do Sistema}
\label{sec:escopo}
O sistema será um aplicativo móvel que servirá como plataforma de comunicação assíncrona entre médicos de UBSs, docentes especialistas e estudantes de medicina da UFJF. O sistema permitirá que médicos de UBS registrem dúvidas clínicas detalhadas, anexando informações relevantes, e as enviem para a especialidade adequada. O sistema encaminhará a dúvida ao docente supervisor e seus estagiários, que poderão elaborar uma resposta colaborativamente para ser enviada de volta ao médico solicitante.

\textbf{Fora do Escopo:} O sistema não incluirá funcionalidades de teleconsulta por vídeo em tempo real, prescrição eletrônica ou agendamento de consultas.

\section{Metas de Design}
\label{sec:metas_design}
As metas de design do sistema são:
\begin{itemize}
    \item \textbf{Reduzir o tempo de resposta:} Diminuir o tempo necessário para que um médico de UBS obtenha uma segunda opinião qualificada.
    \item \textbf{Aumentar a eficiência (Meta Cognitiva):} Projetar a interface para que a tarefa de registrar uma dúvida ou responder a uma consulta exija o mínimo de carga cognitiva, apresentando as informações de forma clara e guiando o usuário passo a passo.
    \item \textbf{Melhorar a segurança (Meta de Usuário):} Garantir a segurança e a confidencialidade dos dados dos pacientes, em conformidade com a LGPD, para que os profissionais se sintam seguros ao usar a ferramenta.
    \item \textbf{Promover o aprendizado (Meta de Usuário):} Criar um ambiente onde os estudantes possam participar ativamente da resolução de casos clínicos reais, visualizando o processo de diagnóstico do especialista.
\end{itemize}

\section{Caracterização dos Usuários e dos Stakeholders}
\label{sec:caracterizacao_usuarios}
Com base na pesquisa apresentada no Capítulo 1, os seguintes usuários e stakeholders foram identificados:

\textbf{Usuários (Papéis Primários):}
\begin{itemize}
    \item \textbf{Médico Generalista (UBS):} Interage com o sistema para criar e enviar dúvidas, acompanhar o status e receber respostas.
    \item \textbf{Docente Especialista:} Interage para receber, analisar e responder às dúvidas, em colaboração com os estudantes.
    \item \textbf{Estudante de Medicina:} Interage para visualizar as dúvidas e auxiliar o docente na elaboração da resposta.
\end{itemize}

\textbf{Stakeholders:}
\begin{itemize}
    \item \textbf{Usuários Primários:} Listados acima.
    \item \textbf{Paciente:} Principal beneficiado com a melhoria da qualidade e agilidade do diagnóstico, embora não interaja com o sistema (usuário secundário).
    \item \textbf{Gestores da Saúde (UBS e Universidade):} Interessados na otimização de recursos e na melhoria dos indicadores de saúde e ensino.
\end{itemize}

\section{Requisitos Funcionais}
\label{sec:requisitos_funcionais}
Abaixo estão especificados os principais requisitos funcionais:

\begin{longtable}{|p{0.25\linewidth}|p{0.7\linewidth}|}
    \caption{Especificação do Requisito Funcional RF01: Efetuar Login} \label{tab:rf01} \\
    \hline
    \multicolumn{2}{|l|}{\textbf{RF01: Efetuar Login}} \\
    \hline
    \endfirsthead
    \multicolumn{2}{c}%
    {{\bfseries\tablename\ \thetable{} -- continuação}} \\
    \hline
    \multicolumn{2}{|l|}{\textbf{RF01: Efetuar Login}} \\
    \hline
    \endhead
    \hline \multicolumn{2}{r}{{\textit{Continua na próxima página}}} \\
    \endfoot
    \hline
    \endlastfoot
    \textbf{Stakeholder(s):} & Médico Generalista, Docente Especialista, Estudante de Medicina. \\
    \hline
    \textbf{Pré-condição(ões):} & O usuário deve possuir um cadastro prévio e ativo no sistema. \\
    \hline
    \textbf{Descrição:} & O usuário informa suas credenciais (ex: CRM ou e-mail) e senha. Ao pressionar "Entrar", o sistema valida as credenciais. \\
    \hline
    \textbf{Pós-condição(ões):} & Se as credenciais forem válidas, o sistema direciona o usuário para sua tela inicial personalizada de acordo com seu perfil. Caso contrário, exibe uma mensagem de erro. \\
    \hline
\end{longtable}

\begin{longtable}{|p{0.25\linewidth}|p{0.7\linewidth}|}
    \caption{Especificação do Requisito Funcional RF02: Registrar Dúvida Clínica} \label{tab:rf02} \\
    \hline
    \multicolumn{2}{|l|}{\textbf{RF02: Registrar Dúvida Clínica}} \\
    \hline
    \endfirsthead
    \multicolumn{2}{c}%
    {{\bfseries\tablename\ \thetable{} -- continuação}} \\
    \hline
    \multicolumn{2}{|l|}{\textbf{RF02: Registrar Dúvida Clínica}} \\
    \hline
    \endhead
    \hline \multicolumn{2}{r}{{\textit{Continua na próxima página}}} \\
    \endfoot
    \hline
    \endlastfoot
    \textbf{Stakeholder(s):} & Médico Generalista, Docente Especialista, Estudante de Medicina, Paciente. \\
    \hline
    \textbf{Pré-condição(ões):} & O Médico Generalista deve estar autenticado no sistema. \\
    \hline
    \textbf{Descrição:} & O médico inicia o fluxo de registro de dúvida. O sistema o guia a preencher campos estruturados: (1) dados demográficos do paciente (anonimizados), (2) especialidade da dúvida (ex: Cardiologia), (3) descrição do caso clínico, (4) campo para anexar arquivos (ex: exames, fotos). Ao finalizar, o médico submete a dúvida. \\
    \hline
    \textbf{Pós-condição(ões):} & A dúvida é salva no sistema com o status "Enviada" e encaminhada automaticamente para a fila do docente especialista correspondente. O médico solicitante pode visualizar a dúvida em seu histórico. \\
    \hline
\end{longtable}

\begin{longtable}{|p{0.25\linewidth}|p{0.7\linewidth}|}
    \caption{Especificação do Requisito Funcional RF03: Responder Dúvida} \label{tab:rf03} \\
    \hline
    \multicolumn{2}{|l|}{\textbf{RF03: Responder Dúvida}} \\
    \hline
    \endfirsthead
    \multicolumn{2}{c}%
    {{\bfseries\tablename\ \thetable{} -- continuação}} \\
    \hline
    \multicolumn{2}{|l|}{\textbf{RF03: Responder Dúvida}} \\
    \hline
    \endhead
    \hline \multicolumn{2}{r}{{\textit{Continua na próxima página}}} \\
    \endfoot
    \hline
    \endlastfoot
    \textbf{Stakeholder(s):} & Docente Especialista, Estudante de Medicina, Médico Generalista. \\
    \hline
    \textbf{Pré-condição(ões):} & O Docente ou Estudante deve estar autenticado. Deve existir uma dúvida na sua fila com o status "Enviada". \\
    \hline
    \textbf{Descrição:} & O docente seleciona uma dúvida de sua lista. Ele visualiza todos os dados e anexos. Há um campo de texto para ele e o estagiário discutirem e elaborarem a resposta. Após a redação final pelo docente, ele submete a resposta. \\
    \hline
    \textbf{Pós-condição(ões):} & A resposta é enviada ao médico solicitante. O status da dúvida é alterado para "Respondida". O médico solicitante recebe uma notificação. \\
    \hline
\end{longtable}

\section{Requisitos Não Funcionais}
\label{sec:requisitos_nao_funcionais}
Os requisitos não funcionais do sistema incluem:
\begin{itemize}
    \item \textbf{Usabilidade:} O sistema deve ser de fácil aprendizado, permitindo que um usuário de primeira viagem consiga registrar uma dúvida em menos de 5 minutos.
    \item \textbf{Segurança:} Toda a comunicação e armazenamento de dados deve utilizar criptografia de ponta a ponta. O sistema deve estar em conformidade com a LGPD e as normas do CFM para telemedicina.
    \item \textbf{Desempenho:} As telas do aplicativo devem carregar em no máximo 3 segundos em uma conexão 4G. O envio de uma dúvida com anexos de até 10MB deve ser concluído em menos de 30 segundos.
    \item \textbf{Disponibilidade:} O sistema deve ter uma disponibilidade de 99,8\% (uptime), com janelas de manutenção planejadas e comunicadas previamente.
    \item \textbf{Comunicabilidade:} A interface deve comunicar claramente o estado do sistema a todo momento (ex: "Enviando dúvida...", "Resposta recebida"), utilizando um feedback claro e imediato para as ações do usuário.
\end{itemize}

\section{Restrições do Sistema}
\label{sec:restricoes}
As restrições do sistema são:
\begin{itemize}
    \item \textbf{Plataforma Operacional:} O sistema será desenvolvido exclusivamente para a plataforma Mobile (Android e iOS).
    \item \textbf{Legislação:} O projeto deve seguir rigorosamente a Lei Geral de Proteção de Dados (LGPD - Lei nº 13.709/2018) e as resoluções do Conselho Federal de Medicina (CFM) sobre telemedicina.
    \item \textbf{Integração:} Nesta primeira versão, o sistema não terá integração com sistemas de prontuário eletrônico de terceiros. A inserção de dados será manual.
    \item \textbf{Idioma:} O único idioma suportado será o Português (Brasil).
\end{itemize}

% --- ELEMENTOS PÓS-TEXTUAIS ---
\postextual

% --- BIBLIOGRAFIA ---
\bibliography{referencias}

\end{document}